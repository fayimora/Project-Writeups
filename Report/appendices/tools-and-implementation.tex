\chapter{Tools and Implementation}
\label{cha:tools-and-implementation}

The dataset was gathered with a \textbf{Node.js} script\footnote{Node.js is a platform used for
building fast and scalable network applications}. The script watched Twitter's public stream and
saved tweets matching a defined pattern to a \textbf{MongoDB} database\footnote{MongoDB is a NoSQL
database with a document-oriented architecture}. The format of tweet objects gotten from Twitter
is the same for a document in MongoDB\@. MongoDB has also been proven to be fast and reliable
which is why it was selected as our database of choice.

The data labelling application was built with \textbf{Express.js}. Express is a web application
framework built on Node.js. It provides a robust set of features for building web applications.

For data classification, we used
scikit-learn\footnote{\url{http://scikit-learn.org/stable/index.html}} \citep{scikit-learn}.
scikit-learn is a machine learning library built in the Python programming language. It provides
algorithms for Classification, Regression, Clustering Dimensionality reduction and Model Selection.
We used its naive Bayes classifier to filter out irrelevant tweets. We also used its Model Selection
module to run grid search in \Sectionref{sec:exhaustive-grid-search}.

For topic modelling, we used gensim\footnote{\url{http://radimrehurek.com/gensim/}}
\citep{rehurek_lrec}. From its website, gensim is ``topic modelling for humans''. It provides
modules for Latent Dirichlet Allocation(LDA) and
others\footnote{\url{http://radimrehurek.com/gensim/apiref.html}}. Our initial attempt at detecting
topic with was done with MALLET\footnote{a Machine Learning for Language Toolkit. See
\url{http://mallet.cs.umass.edu/}} \citep{McCallumMALLET}. The main reasons for selecting gensim
over MALLET are:
\begin{enumerate}
  \item It provides support for online LDA(\Sectionref{sec:bg-lda})
  \item It is very flexible as opposed to MALLET which is highly optimised.
  \item It provides a python API which is simpler than the Java API provided by MALLET.
\end{enumerate}

The plots in Chapter~\ref{cha:data-classification} were drawn with
matplotlib\footnote{\url{http://matplotlib.org/}}, a 2D plotting library \citep{hunter-2007} and the
word clouds in Chapter~\ref{cha:topic-modelling} were generated by Jason Davies's word cloud
generator\footnote{\url{https://www.jasondavies.com/wordcloud}}.

% (pages~\pageref{fig:30-topics-cloud} and~\pageref{fig:40-topics-cloud})
