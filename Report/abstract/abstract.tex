
\addcontentsline{toc}{chapter}{Abstract}

\begin{abstract}
Organisations today are always looking for ways to improve their customers' satisfaction. In order
to do this, they request feedback from these customers via surveys and this is sometimes an
inconvenience to the customer. One drawback with using surveys is that a large proportion of the
questions are left unanswered.

In this study, we propose a way of getting this feedback using social media. Social platforms like
Facebook and Twitter generate an enormous amount of data on a daily basis. One of the problems with
using data from social media is that it contains a lot of irrelevancy. To solve this problem, we
train a Na\"{i}ve Bayes Classifier to automatically filter out irrelevant data from our dataset.
The performance of the classifier is also evaluated and the main metric used is the Area Under the
Receiver Operating Characteristic Curve(AUC). After careful pre-processing and parameters
optimization, we were able to achieve 0.85 AUC(85\%).

Finally, we attempt to detect themes in our relevant dataset using Latent Dirichlet Allocation. We
build and analyse a topic model that generates 30 topics. We empirically evaluate the result of the
model by making some assumptions about the generated topics and validating our assumptions by
analysing the tweets categorised under that topic. We also build a new topic model(on the same
dataset) with 40 topics. We then search for similarities in the 30-topics model and 40-topics model.
In summary, our evaluation of the topic models show that it is possible to detect themes in social
data.

\end{abstract}



% People sometimes use
% these platforms as an avenue to express their thoughts about products they use. They have
% discussions with each other about these products and make comparisons. We make use of Apple
% Incorporated and its products as a case study.
