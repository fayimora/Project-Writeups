
\addcontentsline{toc}{chapter}{Abstract}

\begin{abstract}

Every organisation out there today is constantly looking for ways to improve customer satisfaction.
Technology firms like Apple, Samsung and Google want to know if their software/hardware products
meets the consumers needs. Merchandise retailers like Walmart and Tesco are constantly trying to
make sure they are serving the right products in the right quantity and at the right price. Startups
continuously evaluate their products to measure the probability of the company being successful
sometime in the future. Postal services like Royal Mail are very interested in how their services
are doing and what their customers despise most so they can improve. The big question is how do they
do this? %How does Royal Mail know what its customers think about the price of its services?

Social platforms like Facebook and Twitter generate an enormous amount of data on a daily basis.
People sometimes use these platforms as an avenue to express their thoughts about products they use.
They have discussions with each other about these products and make comparisons.

In this study, we will be making use of Apple Incorporated as a case study. We start by mining Apple
related data from Twitter and then we proceed to filtering this data into what is relevant and what
isn't. Once we have our relevant data, we will use a mixture of Machine Learning and Natural
Language Processing techniques to find common topics in the data. Furthermore, we will analyze the
sentiments of the data and investigate how it correlates with the topics. Lastly, we will evaluate
the techniques applied to determine which ones work best and why.

\end{abstract}
