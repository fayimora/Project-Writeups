
\chapter{Background Theory}

\label{ch:background}

\section{Introduction}
\label{sec:background_introduction}
% In this chapter, we explain some concepts and take a look at some background work related to this
% study. Firstly, we review some research in the use of the naive Bayes Classifier for text
% classification. Then we take a look at some background in topic modelling, specifically the Latent
% Dirichlet Allocation

Automatic Text Classification or Text Categorization is a rapidly growing field in Machine Learning
and Natural Language Processing. This mainly due to the amount of electronic data we currently
generate. The main task is to assign one or more classes to a given text document. Applications of
text classification include \textit{Email Spam Detection} and \textit{Language
Detection}. The former involves trying to distinguish spam emails from legitimate ones while the
latter involves the identification of the language a document was written in.

However, this study makes use of classification techniques for data filtration (removing irrelevant
documents from a list of documents, similar to spam filtering), topic modelling (extracting topics
from a list of documents) and sentiment analysis (predicting the sentiment of the author of a
document). This chapter explains a few background concepts and reviews some relevant research
previously done in this area.

\section{Naive Bayes Classifier}
\label{sec:bg_text_classification}
The naive Bayes classifier is the simplest classifier that can be used and this is due to the fact
that is is based on simple Bayes Theorem. This means it is a probabilistic classifier which
assumes that all features of the documents are independent of each other.

Bayes theorem states that the probability of $A$ given $B$ is the probability of $B$ given $A$ times
the probability of $A$ divided by the probability of $B$. Mathematically, this is written as:
\begin{equation}
  p(A|B) = \frac{p(B|A)p(A)}{p(B)}
\end{equation}

Applying this logic to text classification, the probability that a document $d_i \in D$ belongs to a
class $c$ is denoted as:
\begin{equation}
  p(c|d_i) = \frac{p(d_i|c)p(c)}{p(d_i)}
\end{equation}

Although other techniques like Maximum Entropy, Random Forests or Support Vector Machines tend to
perform better, a naive Bayes classifier will require less memory and CPU cycles. It is also
computationally less complex and simpler to implement. With regard to
performance,~\cite{huangLuLing2003} proved using multiple datasets from~\cite{blakeMerz1998} that
the naive Bayes classifier in many cases performs as good as other complex
classifiers and~\cite{zhang2004} goes further to explain why it performs well. Other studies have also
found Bayesian classifiers to be effective without being affected by its simple independence
assumption~\cite{langley1992analysis,manning2008}.

\section{Topic Modelling}
\label{sec:bg_topic_modelling}

% \section{Sentiment Analysis}
% \label{sec:bg_sentiment_analysis}

\section{Model Evaluation}
\label{sec:bg_model_evaluation}

