
\chapter{Conclusion}
\label{ch:conclusions}

\section{Summary of Report Achievements}
In Chapter~\ref{cha:data-classification}, we were able to create a model to classify tweets into
relevant and non relevant groups. We learnt that accuracy is not a good measure for a classifier's
performance. As a result we decided to use the Area Under the Receiver Operating Characteristic
Curve(AUC) as an evaluation metric. We also used k-fold cross validation to evaluate the performance
of our classifier on an unseen dataset. The average AUC of our
initial(\Sectionref{sec:training-initial-classifier}) and best
model(\Sectionref{sec:exhaustive-grid-search}) were 0.74 and 0.85, respectively. This means we
achieved a 13.8\% increase in AUC.

In other to create our initial training set, we used a web application which displayed a number of
tweets and options to classify them into relevant and irrelevant groups( \Figref{fig:labeller}). The
application was created for this project but can be used in other similar projects. The only
requirement is that tweets are stored in a MongoDB database. It can also be extended easily to
support extra features if required.

In Chapter~\ref{cha:topic-modelling}, we created two topic models, one with 30 topics and the other
with 40 topics. We empirically analysed and evaluated some of the topics generated by the 30 topics
model. We discovered that the topic model was able to correctly place the right tweets under the
right topics. We also learnt that LDA can sometimes merge two or three topics into one(
\Sectionref{sec:topic-9}, \Sectionref{sec:topic-13}, \Sectionref{sec:topic-27}).


\section{Applications}

Applications.
% Sporting events like the Olympics or World Cup.
% Companies offering products nd services like Apple, Google and Microsoft.


\section{Future Work}

Future Work.
% More balanced training dataset
% Compare and contrast Naive Bayes with SVM
% Semi supervised approach to LDA.
% Investigate statistical ways of evaluating topic models like harmonic mean.
