
\chapter{Introduction}
% In this chapter, we discuss the aims and objectives of this study.


\section{Motivation and Objectives}
The main aim of this project is to investigate the use of Machine Learning and Natural Language
Processing techniques on social data. Every organisation today is continuously search for new ways
to get feedback from their clients/users. Current ways of achieving this include
\textbf{Surveys} (questionaires or interviews) and \textbf{Focus Groups}.

Surveys have the advantage being very easy to create and distribute. There are also a variety of
tools to help with this. Some of them include SurveyMonkey\footnote{https://www.surveymonkey.com/},
Google Docs\footnote{https://drive.google.com}. Unfortunately, Surveys also have a few unpleasant
drawbacks like time consumption and labour intensity. It can also be difficult to encourage
partcipants to respond. Nonetheless, the main drawback to using Surveys is that the some questions
are left unanswered while the answers given may not reflect the truthful sentiments of the
participant.~\cite{DonaldBRubin1987} concurs with this and he goes on to discuss how this problem
can be solved (to a certain extent) with imputation\footnote{Imputation is the process of inferring
plausible values for missing entries}.~\cite{BobEHayes2008} also agrees with this point of view and
suggests the use of well designed leading questions to put the participant in the right frame of
mind. For instance, a leading question like ``\textit{How likely will you recommend our service to
friends?}'' gets the participant thinking about recommendations. While the above solutions might
work, they also have the same drawbacks as the original problem. Imputation can be very time
consuming, labour intensive and error prone while the use of leading questions fails to solve the
problem of unanswered questions.

Unfortunately, interviews and focus groups also suffer from false answers due to the fact that they
are not anonymous. This means that the participants, in the face of an interviewer, try to be
linient in other not to sound too negative. This could also somtimes be due to the fact that
participation in the interview/focus group has been incentivised with money or desirable items.

Ideally, the next question we should be asking is ``\textit{How can we get reviews and thoughts
about our products and services from customers, voluntarily?}''


\section{Why Twitter?}
Twitter is a social micro-blogging platforms where users can share messages in 140 characters.
It also allows its users to follow each other. This means, if person A follows person B, A will see
public posts from B. These messages are usually referred to as tweets.

Tweets are capped to 140 characters and can contain text, links or a combination of both. They are
usually related to either an event, interests or just personal opinion. Facebook posts are mostly
always well thought out and each post might include multiple topics. Tweets on the other hand are
usually written at the speed of thought. This makes it a good source of data.

According to Mashable, DOMO, a Business Intelligence company paired up with Column Five Media to
create an infographic\footnote{http://mashable.com/2012/06/22/data-created-every-minute/} about the
web back in 2012. It showed that Twitter at the time received around 100,000 tweets per minute.

Finally, Twitter's data is open compared to other social platforms like Facebook. This means
developers are free to tap into this wealth of data in almost real time. This makes Twitter a
perfect source for our data.


\section{Statement of Originality}

Statement here.

