
\chapter{Introduction}
% In this chapter, we discuss the aims and objectives of this study.


\section{Motivation and Objectives}
The main aim of this project is to investigate the use of Machine Learning and Natural Language
Processing techinques on social data. As we see later in this study, most if not all of these
techinques have already been tested on textual data by other researches but not much has been done
on social data.

\section{Why Twitter?}
Twitter is a social micro-blogging platforms where users can share their thoughts in 140 characters.
It also allows its users to follow each other. This means, if person A follows person B, A will see
public posts from B. These messages are usualy referred to as tweets.

Tweets are capped to 140 characters and can contain text, links or a combination of both. They are
usually related to either an event, interests or just personal opinion. Facebook posts are mostly
always well thought out and each post might include multiple topics. Tweets on the other hand are
usually written at the speed of thought. This makes it a good source of data.

According to Mashable, DOMO, a Business Intelligence company paired up with Column Five Media to
create an infographic\footnote{http://mashable.com/2012/06/22/data-created-every-minute/} about the
web back in 2012. It showed that Twitter at the time received around 100,000 tweets per minute.

Finally, Twitter's data is open compared to other social platforms like Facebook. This means
developers are free to tap into this wealth of data in almost real time. This makes Twitter a
perfect source for our data.


\section{Statement of Originality}

Statement here.

