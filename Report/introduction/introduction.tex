
\chapter{Introduction}
% In this chapter, we discuss the aims and objectives of this study.


\section{Motivation and Objectives}
The main aim of this project is to investigate the use of Machine Learning and Natural Language
Processing techniques on social data. Every organisation today is continuously search for new ways
to get feedback from their clients/users. Current ways of achieving this include
\textbf{Surveys} (questionnaires or interviews) and \textbf{Focus Groups}.

Surveys have the advantage being very easy to create and distribute. There are also a variety of
tools to help with this. Some of them include SurveyMonkey\footnote{https://www.surveymonkey.com/}
and Google Docs\footnote{https://drive.google.com}. Unfortunately, Surveys also have a few
unpleasant drawbacks like time consumption and labour intensity. It can also be difficult to
encourage participants to respond. Nevertheless, the main drawback to using Surveys is that some
questions are left unanswered while the answers given in answered questions may not reflect the
truthful sentiments of the participant.~\cite{DonaldBRubin1987} concurs with this and he goes on to
discuss how this problem can be solved (to a certain extent) with imputation\footnote{Imputation is
the process of inferring plausible values for missing entries}.~\cite{BobEHayes2008} also agrees
with this point of view and suggests the use of well designed leading questions to put the
participant in the right frame of mind. For instance, a leading question like ``\textit{How likely
will you recommend our service to friends?}'' gets the participant thinking about recommendations.
While the above solutions might work, they also have the same drawbacks as the original problem.
Imputation can be very time consuming, labour intensive and error prone while the use of leading
questions fails to solve the problem of unanswered questions.

Unfortunately, interviews and focus groups also suffer from false answers due to the fact that they
are not anonymous. This means that the participants, in the face of an interviewer, try to be
lenient in other not to sound too negative. This could also sometimes be due to the fact that
participation in the interview/focus group has been incentivised with money or desirable items.

Ideally, the next question we should be asking is ``\textit{How can we get the truthful views of our
clients about our products/services?}''. We need to find a way to get this information without
putting any pressure on our clients. Fortunately, the recent surve in the use of social media makes
this very easy. People, more often than not, tend to post their truthful feelings about services
they use on social media. For instance, Person A buys an iPhone today and realises that the Wifi
connectivity is faulty. He/She will most likely post something like ``\texttt{Got an iPhone today
and I cannot use my wifi. \#NotCool}'' on one or more of the available social networking platforms.
From this statement, we can infer that Person A is talking about \textit{the iPhone}, \textit{Wifi}
and \textit{Connectivity}. We could also infer that the sentiment of the user, with respect to those
topics, is \textit{negative}.  The process of discovering abstract topics in text is called
\textbf{Topic Modelling} while the process of discovering sentiments in text is known as
\textbf{Sentiment Analysis} and Chapters~\ref{cha:topic_modelling} and~\ref{cha:sentiment_analysis}
discuss how we can automate this processes, respectively.

\section{Why Twitter?}
Twitter is a social micro-blogging platforms where users can share messages in 140 characters. It
also allows its users to follow each other. This means, if person A follows person B, A will see
public posts from B. These messages are usually referred to as tweets.

Tweets are capped to 140 characters and can contain text, links or a combination of both. They are
usually related to either an event, interests or just personal opinion. Facebook posts are mostly
always well thought out and each post might include multiple topics. Tweets on the other hand are
usually written at the speed of thought. This makes it a good source of data.

According to Mashable, DOMO, a Business Intelligence company paired up with Column Five Media to
create an infographic\footnote{http://mashable.com/2012/06/22/data-created-every-minute/} about the
web back in 2012. It showed that Twitter at the time received around 100,000 tweets per minute.

Finally, Twitter's data is open compared to other social platforms like Facebook. This means
developers are free to tap into this wealth of data in almost real time. This makes Twitter a
perfect source for our data.

\section{Methodology}
\label{sec:methodology}

\section{Statement of Originality}

Statement here.

