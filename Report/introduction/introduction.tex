
\chapter{Introduction}
\label{cha:introduction}

\section{Motivation}
\label{sec:motivation}
Organisations today continuously search for new ways to get feedback from their clients in a bid to
improve customer satisfaction. Technology firms like Apple, Samsung and Google want to know if their
software/hardware products meet their consumers' needs. Merchandise retailers like Walmart and Tesco
are constantly trying to make sure they are serving the right products in the right quantity and at
the right price. Startups continuously evaluate their products to measure the probability of the
company being successful sometime in the future. Postal services like Royal Mail are very interested
in how their services are doing and what their customers despise most so they can improve. Current
ways of achieving this include \textbf{Surveys} (questionnaires or interviews) and \textbf{Focus
Groups}.

Surveys are very easy to create and distribute. There are also a variety of tools to help with this.
Some of them include SurveyMonkey\footnote{https://www.surveymonkey.com/} and Google
Docs\footnote{https://drive.google.com}. Unfortunately, Surveys also have a few unpleasant drawbacks
like time consumption and labour intensity. It can also be difficult to encourage participants to
respond. Nevertheless, the main drawback to using Surveys is that some questions are left unanswered
while the answers given in answered questions may not reflect the truthful sentiments of the
participant.~\citet{rubin1987} concurs with this and he goes on to discuss how this problem can be
solved (to a certain extent) with imputation\footnote{Imputation is the process of inferring
plausible values for missing entries}.~\citet{hayes2008} also agrees with this point of view and
suggests the use of well designed leading questions to put the participant in the right frame of
mind. For instance, a leading question like ``\textit{How likely will you recommend our service to
friends?}'' gets the participant thinking about recommendations. While the above solutions might
work, they have the same drawbacks as the original problem. Imputation can be very time
consuming, labour intensive and error prone while the use of leading questions fails to solve the
problem of unanswered questions.

Unfortunately, interviews and focus groups also suffer from false answers due to the fact that they
are not anonymous. This means that the participants, in the face of an interviewer, try to be
lenient in other not to sound too negative. This could sometimes be due to the fact that
participation in the interview/focus group has been incentivised with money or desirable items.

Ideally, the next question we should be asking is ``\textit{How can we get the truthful views of our
clients about our products and services?}''? We need to find a way to get this information without
putting any pressure on our clients.

\section{Aims and Objectives}
\label{sec:objectives}
The main aim of this project is to investigate other means of getting our data and also, how we can
make use of Machine Learning and Natural Language Processing techniques to make sense of the data.

Fortunately, the recent surge in the use of social media makes the former relatively easy. People,
more often than not, tend to post their truthful feelings about services they use on social media.
For instance, Person A buys an iPhone today and realises that the Wi-Fi connectivity is faulty.
He/She will most likely post something like ``\textit{New iPhone wifi not working \#NotCool}'' on
one or more of the available social networking platforms. From this statement, we can infer that
Person A is talking about \textit{the iPhone}, \textit{Wi-Fi} and \textit{Connectivity}. The process
of discovering abstract topics in text is called \textbf{Topic Modelling}.
Chapter~\ref{cha:topic-modelling} discusses how we can automate this process.

We will try to answer some research questions. They include:
\begin{itemize}
  \item Can we use supervised techniques to accurately classify tweets into what is relevant and
    what is not?
  \item Can we detect themes/topics in our dataset? If yes, are these topics related to Apple Inc in
    any way?
  \item One way to know the preferences of anyone is by knowing their interests. Can we
    profile/group people in terms of their interests?
\end{itemize}


\section{Why Twitter?}
\label{sec:why-twitter}
Twitter is a social micro-blogging platform where users can share messages in 140 characters. It
also allows its users to follow each other. This means, if person A follows person B, A will see
public posts from B. These messages are usually referred to as tweets.

Tweets are capped to 140 characters and can contain text, links or a combination of both. They are
usually related to either an event, interests or just personal opinion. Facebook posts are mostly
always well thought out and each post might include multiple topics. Tweets on the other hand are
usually written at the speed of thought. This makes it a good source of data.

% TODO: Get official and up to date statistics from Twitter
According to Mashable, DOMO, a Business Intelligence company paired up with Column Five Media to
create an infographic\footnote{See \url{http://mashable.com/2012/06/22/data-created-every-minute/}}
about the web back in 2012. It showed that Twitter at the time received around 100,000 tweets per
minute. As at 1st February 2014 Twitter claims to receive 500 million tweets a day\footnote{See
https://about.twitter.com/company}. That is roughly 350,000 tweets per minute which is 3 times the
amount 2 years before. Twitter also claims to have 241 million monthly users.

Finally, Twitter's data is open compared to other social platforms like Facebook. This means
developers are free to tap into this wealth of data in almost real time. This makes Twitter a
perfect source for our data.


% TODO: very much bare bones
\section{Methodology}
\label{sec:methodology}
This study requires social data and the dataset used is gathered from Twitter over a time frame.
The data collected is related to Apple Inc and their products.

With data already gathered, we train a classifier to help filter out as many irrelevant tweets as
possible. We briefly analyse different ways to filter the dataset but eventually settle with using
Na\"{i}ve Bayes Classifier. We also look into different ways of analysing the classifier's
performance and ways it can be improved.

Finally, we attempt to identify topics/themes in the dataset. We briefly look at Latent Semantic
Indexing and why it might not be suitable for our needs. We then look into Latent Dirichlet
Allocation, a common approach to topic modelling and use it to detect topics in our dataset. The
evaluation of topics generated will be analysed empirically. This means we take a topic and make
some assumptions about the semantics of the tweets belonging to that topic. We then analyse the
tweets to confirm the validity of our assumption.

\section{Statement of Originality}

Statement here.

